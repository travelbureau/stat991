%
% This is a borrowed LaTeX template file for lecture notes for CS267,
% Applications of Parallel Computing, UC Berkeley EECS Department.
% Now being used for Penn's STAT 991: Optimization Methods in Machine Learning 
% taught by Weijie Su.  When
% preparing LaTeX notes for this class, please use this template. 
%
% To familiarize yourself with this template, the body contains
% some examples of its use.  Look them over.  Then you can
% run LaTeX on this file.  After you have LaTeXed this file then
% you can look over the result either by printing it out with
% dvips or using xdvi. "pdflatex template.tex" should also work.
%

\documentclass[twoside]{article}
\setlength{\oddsidemargin}{0.25 in}
\setlength{\evensidemargin}{-0.25 in}
\setlength{\topmargin}{-0.6 in}
\setlength{\textwidth}{6.5 in}
\setlength{\textheight}{8.5 in}
\setlength{\headsep}{0.75 in}
\setlength{\parindent}{0 in}
\setlength{\parskip}{0.1 in}

%
% ADD PACKAGES here:
%

\usepackage{amsmath,amsfonts,graphicx}
\usepackage{multirow}

%
% The following commands set up the lecnum (lecture number)
% counter and make various numbering schemes work relative
% to the lecture number.
%
\newcounter{lecnum}
\renewcommand{\thepage}{\thelecnum-\arabic{page}}
\renewcommand{\thesection}{\thelecnum.\arabic{section}}
\renewcommand{\theequation}{\thelecnum.\arabic{equation}}
\renewcommand{\thefigure}{\thelecnum.\arabic{figure}}
\renewcommand{\thetable}{\thelecnum.\arabic{table}}

%
% The following macro is used to generate the header.
%
\newcommand{\lecture}[4]{
   \pagestyle{myheadings}
   \thispagestyle{plain}
   \newpage
   \setcounter{lecnum}{#1}
   \setcounter{page}{1}
   \noindent
   \begin{center}
   \framebox{
      \vbox{\vspace{2mm}
    \hbox to 6.28in { {\bf STAT 991: Optimization Methods in Machine Learning \hfill Spring 2019} }
       \vspace{4mm}
       \hbox to 6.28in { {\Large \hfill Lecture #1: #2  \hfill} }
       \vspace{2mm}
       \hbox to 6.28in { {\it Instructor: #3 \hfill Scribes: #4} }
      \vspace{2mm}}
   }
   \end{center}
   \markboth{Lecture #1: #2}{Lecture #1: #2}

   % {\bf Disclaimer}: {\it These notes have not been subjected to the
   % usual scrutiny reserved for formal publications. They may be
   % distributed outside this class only with the permission of the
   % Instructor.} \vspace*{4mm}
}
%
% Convention for citations is authors' initials followed by the year.
% For example, to cite a paper by Leighton and Maggs you would type
% \cite{LM89}, and to cite a paper by Strassen you would type \cite{S69}.
% (To avoid bibliography problems, for now we redefine the \cite command.)
% Also commands that create a suitable format for the reference list.
\renewcommand{\cite}[1]{[#1]}
\def\beginrefs{\begin{list}%
        {[\arabic{equation}]}{\usecounter{equation}
         \setlength{\leftmargin}{2.0truecm}\setlength{\labelsep}{0.4truecm}%
         \setlength{\labelwidth}{1.6truecm}}}
\def\endrefs{\end{list}}
\def\bibentry#1{\item[\hbox{[#1]}]}

%Use this command for a figure; it puts a figure in wherever you want it.
%usage: \fig{NUMBER}{SPACE-IN-INCHES}{CAPTION}
\newcommand{\fig}[3]{
			\vspace{#2}
			\begin{center}
			Figure \thelecnum.#1:~#3
			\end{center}
	}
% Use these for theorems, lemmas, proofs, etc.
\newtheorem{theorem}{Theorem}[lecnum]
\newtheorem{lemma}[theorem]{Lemma}
\newtheorem{proposition}[theorem]{Proposition}
\newtheorem{claim}[theorem]{Claim}
\newtheorem{corollary}[theorem]{Corollary}
\newtheorem{definition}[theorem]{Definition}
\newenvironment{proof}{{\bf Proof:}}{\hfill\rule{2mm}{2mm}}

% **** IF YOU WANT TO DEFINE ADDITIONAL MACROS FOR YOURSELF, PUT THEM HERE:

\newcommand\E{\mathbb{E}}

\begin{document}
%FILL IN THE RIGHT INFO.
%\lecture{16}{**DATE**}{**LECTURER**}{**SCRIBE**}
\lecture{16}{January 17, 2019}{Weijie Su}{Matthew O'Kelly, Hongji Wei, Yiliang Zhang}
%\footnotetext{These notes are partially based on those of Nigel Mansell.}

% **** YOUR NOTES GO HERE:

% Some general latex examples and examples making use of the
% macros follow.  
%**** IN GENERAL, BE BRIEF. LONG SCRIBE NOTES, NO MATTER HOW WELL WRITTEN,
%**** ARE NEVER READ BY ANYBODY.

\section{Overview}
\label{sec:overview}

This lecture covers the following topics:
\begin{enumerate}
\item Stochastic Gradient Descent
\item Regularized Dual Averaging
\item Judistry-Polyak Averaging
\end{enumerate}

\section{Stochastic Gradient Descent}
Consider the follow optimization problem:
\begin{equation}
	\min f(\theta) = \mathbb{E}f(\theta, z)
\end{equation}
In the setting of stochastic gradient descent (SGD) the optimizor is given access only to a realization of the random function $\mathbb{E} (\theta, z)$; however, the goal of SGD is still to minimize the deterministic function $f(\theta)$. 
In general $\mathbb{E}(\theta, z)$ is a high-dimensional integral which cannot be directly computed. Thus instead the optimizer recieves a noisy realization of the function computed as:
\begin{equation}
	\frac{1}{n}\sum_{i=1}^{n} f(\theta, z_I)
\end{equation}

In what follows we will derive results where $f(\theta)$ is assumed to be convex but not necessarily smooth. In the case where $f(\theta)$ is convex and non-smooth the subgradient is denoted as: 
\begin{equation}
\delta_{\theta} f (\theta,z)
\end{equation}


Update step is given as: 
\begin{equation}
	\theta_{k+1} = \theta_k + \gamma_k \left(\theta_k, z_k\right)
\end{equation}

Where $\theta_k$ is the step-size. 

\section{Regularized Dual Averaging}

Regularized dual averaging (Xiao 2010, Nesterov 2009) is proposed to solve the penalized optimization problem
\begin{align*}
	\min f(\theta) &= \mathbb{E}f(\theta,Z) + \phi(\theta)\\
		&= \frac{1}{n}\sum_{i=1}^n f(\theta,Z_i) + \phi(\theta)
\end{align*}

Similar as SGD problem, we can only see finite i.i.d random samples $Z_1, Z_2, ..., Z_n$ drawn from a certain population. Therefore, instead of minimizing $\mathbb{E}f(\theta,Z) + \phi(\theta)$, we are going to minimize over the surrogate $\frac{1}{n}\sum_{i=1}^n f(\theta,Z_i) + \phi(\theta)$.

Here are some examples of penalty function $f(\theta)$:
\begin{equation*}
	\phi(\theta) = \begin{cases} 
		\lambda\Vert \theta \Vert_1 &\text{$l_1$ penalty (Lasso)}\\
		\lambda\Vert \theta \Vert_2 &\text{$l_2$ penalty (ridge)}\\
		\mathbb{I}_C &\text{indicator function (constraint on convex closed set $C$)}\\
		\sum \lambda_i|\theta|_{(i)} &\text{sorted $l_1$ penalty (Slope)}
	\end{cases}
\end{equation*}

Regularized dual averaging is a method to solve the optimization problem
\begin{equation*}
	\min f(\theta) = \frac{1}{n}\sum_{i=1}^n f(\theta,Z_i) + \phi(\theta)
\end{equation*}

First, RDA takes initialized value $\theta_0$. Similar as SGD, at $k$th step, the algorithm draws $i_k$ uniformly from indices set $\{1,2,...,n\}$ and calculate the gradient with only one sample $Z_{i_k}$:
\[ g_k = \partial f(\theta_k, Z_{i_k}) \]

However, the update rule of RDA is different from SGD, which is
\[ \theta_{k+1} = \arg\min_\theta \left\{ \langle \frac{g_1 + g_2 + ... g_k}{k}, \theta \rangle + \phi(\theta) + \frac{B_k}{k}\cdot\frac{\Vert \theta \Vert^2}{2}  \right\}  \]

The name "dual averging" comes from the special term $\frac{g_1 + g_2 + ... g_k}{k}$ in the above updata rule. $g_1, g_2, ..., g_k$ are gradients thus can be viewed as vectors from the dual space. Here we evaluate the average of these $k$ gradients so we call this method "dual averging".

\textbf{Remarks:}
\begin{enumerate}
	\item Although we need to average over all previous gradients, this algorithm is indeed a online algorithm. Because at the $k$th step we only need to record the average
	\[ \bar{g}_k = \frac{g_1 + g_2 + ... + g_k}{k}  \]
	
	At the next step, the average can be updated by using only $\bar{g}_k$ and $g_{k+1}$
	\[ \bar{g}_{k+1} = \frac{k}{k+1}\bar{g}_k + \frac{1}{k+1}g_{k+1}  \]
	
	\item $\frac{\Vert \theta \Vert^2}{2}$ can be replaced if $f$ is a strongly convex function.
\end{enumerate}

\textbf{Prof. Su's interpretation of RDA:}
To simplify notations, let's assume at $k$th step we observe a data point $Z_k$. Then by a crude derivation,
\begin{align*}
	f(\theta) &= \mathbb{E}f(\theta,Z) + \phi(\theta) \\
	&\approx \frac{1}{k}\sum_{j=1}^k f(\theta,Z_j) + \phi(\theta) \\
	&\approx \frac{1}{k}\sum_{j=1}^k( f(\theta_j,Z_j) + \langle \partial f(\theta_j,Z_j), \theta - \theta_j \rangle + \text{quadratic} ) + \phi(\theta) \\
	&= \langle \frac{1}{k}\sum_{j=1}^k g_j, \theta \rangle + \phi(\theta) + \text{quadratic} + \text{constant}
\end{align*}

Therefore, minimizing $f(\theta)$ is approximatly equivalent to minimizing
\[ \langle \frac{1}{k}\sum_{j=1}^k g_j, \theta \rangle + \phi(\theta) + \text{quadratic}  \]

The next (informal) theorem gives the convergence rate of RDA:
\begin{theorem}
	If $\mathbb{E} f(\theta,Z)$ is convex and $L-Lipstichiz$, let $B_k = C\sqrt{k}$, and denote $\bar\theta_k = \frac{1}{k+1}\sum_{i=0}^k \theta_i$, we have
	\[ \mathbb{E} f(\theta,Z) - f(\theta^\ast) = O(\frac{1}{\sqrt{k}}) \]
\end{theorem}

Although convergence rate of RDA is still $\frac{1}{\sqrt{k}}$, the same as SGD, it has several benifits compared to SGD:
\begin{enumerate}
	\item Solution in RDA can be sparse for $\phi(\theta) = \lambda\Vert\theta\Vert_1$.
	\item Empirical performance of RDA is better than SGD.
	\item Dual averaging can be a trick generalizing to distributed settings (Duchi, Agarval, Wainwright 2012).
	\item (this is like a remark, not a comparison between RDA and SGD.) SVRG(2013) is better than dual averaging. This comparison is not that fair because SVRG is not an online method. But SVRG is more popular in industry.
\end{enumerate}
\section{Judistry-Polyak Averaging}


%\section{Some theorems and stuff}
% % Don't be this informal in your notes!
%
%We now delve right into the proof.
%
%\begin{lemma}
%This is the first lemma of the lecture.
%\end{lemma}
%
%\begin{proof}
%The proof is by induction on $\ldots$.
%For fun, we throw in a figure.
%%%%NOTE USAGE !
%\fig{1}{1in}{A Fun Figure}
%
%This is the end of the proof, which is marked with a little box.
%\end{proof}
%
%\subsection{A few items of note}
%
%Here is an itemized list:
%\begin{itemize}
%\item this is the first item;
%\item this is the second item.
%\end{itemize}
%
%Here is an enumerated list:
%\begin{enumerate}
%\item this is the first item;
%\item this is the second item.
%\end{enumerate}
%
%Here is an exercise:
%
%{\bf Exercise:}  Show that ${\rm P}\ne{\rm NP}$.
%
%Here is how to define things in the proper mathematical style.
%Let $f_k$ be the $AND-OR$ function, defined by
%
%\[ f_k(x_1, x_2, \ldots, x_{2^k}) = \left\{ \begin{array}{ll}
%
%	x_1 & \mbox{if $k = 0$;} \\
%
%	AND(f_{k-1}(x_1, \ldots, x_{2^{k-1}}),
%	   f_{k-1}(x_{2^{k-1} + 1}, \ldots, x_{2^k}))
%	 & \mbox{if $k$ is even;} \\
%
%	OR(f_{k-1}(x_1, \ldots, x_{2^{k-1}}),
%	   f_{k-1}(x_{2^{k-1} + 1}, \ldots, x_{2^k}))	
%	& \mbox{otherwise.} 
%	\end{array}
%	\right. \]
%
%\begin{theorem}
%This is the first theorem.
%\end{theorem}
%
%\begin{proof}
%This is the proof of the first theorem. We show how to write pseudo-code now.
%%*** USE PSEUDO-CODE ONLY IF IT IS CLEARER THAN AN ENGLISH DESCRIPTION
%
%Consider a comparison between $x$ and~$y$:
%\begin{tabbing}
%\hspace*{.25in} \= \hspace*{.25in} \= \hspace*{.25in} \= \hspace*{.25in} \= \hspace*{.25in} \=\kill
%\>{\bf if} $x$ or $y$ or both are in $S$ {\bf then } \\
%\>\> answer accordingly \\
%\>{\bf else} \\
%\>\>    Make the element with the larger score (say $x$) win the comparison \\
%\>\> {\bf if} $F(x) + F(y) < \frac{n}{t-1}$ {\bf then} \\%
%\>\>\> $F(x) \leftarrow F(x) + F(y)$ \\
%\>\>\> $F(y) \leftarrow 0$ \\
%\>\> {\bf else}  \\
%\>\>\> $S \leftarrow S \cup \{ x \} $ \\
%\>\>\> $r \leftarrow r+1$ \\
%\>\> {\bf endif} \\
%\>{\bf endif} 
%\end{tabbing}
%
%This concludes the proof.
%\end{proof}
%
%
%\section{Next topic}
%
%Here is a citation, just for fun~\cite{CW87}.

\section*{References}
\beginrefs
\bibentry{RM51}{\sc Robbins, Herbert} and {\sc Sutton Monro}, 
"A stochastic approximation method."
{\it The annals of mathematical statistics},
(1951): 400-407.
\endrefs

\beginrefs
\bibentry{BCN18}
	{\sc Bottou, L{\'e}on} and {\sc Curtis, Frank E} and {\sc Nocedal, Jorge},
	"Optimization methods for large-scale machine learning."
	{\it Siam Review},
	(2018): 233:311.
\endrefs

\beginrefs
\bibentry{SZ16}{\sc Su, Weijie} and {\sc Yuancheng Zhu}, 
"Uncertainty Quantification for Online Learning and Stochastic Approximation via Hierarchical Incremental Gradient Descent." 
{\it arXiv preprint arXiv:1802.04876}
\endrefs

\beginrefs
\bibentry{PJ92}{\sc Polyak, Boris T.} and {\sc Anatoli B. Juditsky}, 
"Acceleration of stochastic approximation by averaging."  
{\it SIAM Journal on Control and Optimization}
30.4 (1992): 838-855.
\endrefs


% **** THIS ENDS THE EXAMPLES. DON'T DELETE THE FOLLOWING LINE:

\end{document}






%%% Local Variables:
%%% mode: latex
%%% TeX-master: t
%%% End:
